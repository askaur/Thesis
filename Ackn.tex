\begin{center}
{\large \bf{\em Acknowledgements}}
\end{center}
\begin{comment}
\vspace*{0.5cm}
{ \em

I take this opportunity to extend my sincere gratitude and appreciation to all those who contributed in many ways to the completion of this Ph.D. thesis. 

To commence with, I pay my obeisance to Almighty GOD to have bestowed upon me good health, courage, inspiration, zeal and the light.

At this moment of accomplishment, first of all I gratefully express my sincere gratitude to my supervisor Prof. Manjit Kaur for her invaluable assistance, motivation, continuous support and blessings. She inspired me greatly to join Ph.D. and under her guidance, I have learned a lot. With her priceless support, I successfully overcame many difficulties. Her willingness to motivate me contributed tremendously to this thesis. Apart from academia, she is as a very good and friendly human being whose presence is highly enjoyable. I would certainly love to be in touch with her in longer run of life.}

I am also extremely indebted to Dr. Klaus Rabbertz for his valuable advice, constructive criticism and extensive discussions. His enthusiasm, simplicity and commitment to physics kept me motivated at every step of research work. His 
me.

It is my great pleasure to acknowledge the assistance and help of Prof.
Klaus Rabbertz, due to whose outstanding commitment especially in the eld of QCD
related physics, I could attain so much knowledge in the subject. It was a great experience
working with him. He always kindly answered all sorts of questions and remained calm in
every situation. His enthusiasm, simplicity and commitment to science deeply impressed
me.


An experiment as complicated as CMS, cannot be successfully exe-
cuted without the dedicated and knowledgeable work of many people. I wish to thank the
members of the CMS Collaboration, who contributed to this analysis in manifold ways, for
their hospitality and giving me a fantastic opportunity of researching in an international
community.


I am highly obliged to Prof. C.S. Aulakh Chairperson of the Department of Physics, Panjab University, Chandigarh, for his helping hand in providing a good environment and adequate facilities to complete this project successfully.

I feel deeply indebted to my parents for their unconditional love, understanding, support, strength, blessings and for everything.

I am greatly thankful to my well wishers, my friends Anu, Sohail, Junaid, Mandeep and my classmates for all that they meant to me during the crucial times of the completion of my project.

I owe my very special thanks to Mr. Inderpal Singh, Mrs. Prabhdeep Kaur, Mrs. Manuk Z. Mehta, Ms. Ritu Aggarwal, Ms. Ruchi Gupta, Ms. Monika Mittal and Ms. Genius, the members of high energy physics Laboratory, Dept. of Physics, P.U. for helping me out from all kinds of problems with their constant support, suggestions and encouragement. 


First of all, I would like to thank Prof. Dr. Hannes Jung for teaching me how to do research in a very
professional and systematic way; thanks for giving me the opportunity to work in this exciting discipline and
to develop my own physics interests and research lines. He has conveyed me a deep passion towards this field
of science and taught me the right approach to investigation and knowledge. I really appreciate how he leads
his working group and I’m very proud and happy to be a part of it.
I would also like to thank my second advisor Prof. Dr. Pierre Van Mechelen for accepting me in the CMS
group at the University of Antwerp and leading my research during my staying there: I really benefitted from
fruitful discussions with him, by getting precious insights for my analysis and significant improvements for
my thesis during the correction phase. Special thanks also to Prof. Dr. Johannes Haller, Prof. Dr. Nick Van
Remortel and Prof. Dr. Dieter Horns for promptly accepting to be referees of my thesis and disputation.
Thanks to Sarah Van Mierlo for her huge patience and kindness in taking care of the bureaucratic part of
my joint PhD and the effort dedicated for making it possible.

And last but not least. Thanks to Jennifer, my anchor and lifemate: she has kindly taken care of me during
stressful periods and enjoyed with me joyful moments. Staying with her is always the best choice of spend-
ing my time. I thank her for being as she is and for letting the best part of me come out.



Financial support by the Helmholtz Alliance “Physics at the Terascale” and
the German Ministry of Education and Research is gratefully acknowledged.}


\\[2cm]
{November 15, 2012} \hspace*{8.cm}
{\em (Anterpreet Kaur)}
\end{comment}
