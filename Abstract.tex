\begin{center}
{\bf \huge Abstract}
\end{center}

High-energy collisions in particle physics experiments produce complex events with large number of particles. An approach to the analysis of these events involves a clustering procedure known as the jet finder algorithm to identify the clusters of particles called jets. The dominant process in Quantum Chromodynamics (QCD) is jet production by scattering of the elementary quark and gluon constituents of the incoming hadron
beams. These jets have a direct connection with the underlying particle physics reaction. Understanding jet structure is the motivation for the present analysis.

In this dissertation, we present the Monte Carlo results on the jet substructure by defining an observable called subjet multiplicity. The subjet multiplicity is defined as the number of subjets that can be resolved within a jet. It is useful to discriminate between quark and gluon jets. The quarks and gluons have different coupling strengths as they possess different color charges. The color factor ratio C$_{A}$/C$_{F}$ = 2.25, with color factors C$_{A}$ = 3, C$_{F}$ = 4/3, indicates that a gluon radiates more than a quark. So the number of subjets within a gluon jet are more than that in quark jet. In the present work, we have used subjet multiplicity to compare  gluon jets to quark jets by ratio r = $\frac{<M_{g}> - 1}{<M_{q}> - 1}$, where $<M_{g}>$ and $<M_{q}>$ are the average subjet multiplicities in gluon and quark jets respectively. The value of r = 1 implies that there is no difference between gluons and quarks whereas the value other than 1 expresses the differences between them. The jet substructure is sensitive to parton showering processes, hence they provide a good test of Monte Carlo
event simulation programs.

In leading order (LO) QCD, there are two partons in the initial and final states
of the elementary process. So a sample of dijet events is selected in the present studies. 
In dijet events, the two jets leading in transverse momentum (P$_{T}$) can be
associated to the two partons at the leading order (LO)
in perturbative QCD. A jet is associated with the energy and momentum of each
final state parton. The internal structure of these jets
is expected to depend mainly on the type of primary parton,
i.e. either (anti-) quark or gluon, from which they
originated.

Jet clustering algorithms have become an indispensable tool for the analysis of hadronic
final states in proton-proton (p-p) collisions. The jets are defined by the $k_{T}$ sequential recombination algorithm and the Cambridge/Aachen sequential recombination algorithm  with a jet size defined in terms of cone radius, $R=0.6$ and $R=1.0$ respectively. 
Subjets are resolved down to a cutoff of $y_\mathrm{cut} = 10^{-3}$ in $k_{T}$ algorithm whereas in Cambridge/Aachen, they are resolved to dcut = $(\frac{Rsub}{R})^{2}$ with Rsub = 0.5. The jets are selected with $p_{T}$>100 GeV. The similar samples of jets are compared at center of mass energies $\sqrt{s}$ = 7 TeV and 10 TeV. At LHC, the study of pp collosions at 7 TeV center of mass energy is possible and to study the collisions at 10 TeV will be possible after the shutdown in 2013. So when the data will be available, the results of this study at generator level will be useful to determine the subjet multiplicities and the C$_{A}$/C$_{F}$ ratio. 

In the present work, we determine the effective value of C$_{A}$/C$_{F}$ ratio to be 1.76 $\pm$ 0.66 with k$_{T}$ algorithm and 1.74 $\pm$ 0.69 with Cambridge algorithm. These values are in agreement with the previously measured and published values from various experiments, as tabulated in section 5.2.
  

\newpage
