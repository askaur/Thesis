\chapter{Introduction}
\label{chap:Introduction}

Particle physics deals with the study of the basic constituents of matter and the forces governing the interactions among them. The Standard Model (SM) is the most accepted theory describing the nature and properties of the fundamental particles and their interactions. The elementary particles leptons and quarks, known as fermions, interact through the exchange of the gauge bosons. The gauge bosons acquire masses in the process of electroweak symmetry breaking whereas the masses of the fermions are generated through Yukawa interactions with the field associated to the scalar Higgs boson. The gauge bosons are the mediators of the four fundamental forces of interaction existing in nature : the electromagnetic force, the strong force, the weak force and the gravitational force. Quantum Choromodynamics (QCD) is the theory of the strong interactions between the quarks mediated by the massless gluons. The quarks and gluons, together known as partons, have a peculiar property of ``color'' charge. Due to confinement property of QCD, the quarks cannot exist freely in nature but bind themselves into colorless particles called hadrons such as protons and neutrons together known as nucleons, pions etc. The structure and the properties of sub-atomic particles can be explored by first accelerating them using particle accelerators and then colliding at very high energies. The end products of these collisions are recorded in the particle detectors constituting the real data. These data sets are analyzed in details to reveal the structure and characteristic properties of the fundamental particles.

To search for the very rare particles, to investigate the physics beyond SM and to explore the regime of undiscovered physical laws, the particle accelerators have become bigger and complex over the past few decades. The Large Hadron Collider (LHC) is one of the biggest and the most powerful particle collider in which the protons are accelerated and collided at extremely high center-of-mass energies to probe their internal structure and the parton distribution functions (PDFs). The PDFs give the probability to find a parton at an energy scale Q carrying a fractional momentum $x$ of the proton. Since the proton is not elementary and is made up of partons, the proton-proton (pp) collisions are viewed as interactions between their constituent partons. The final products of the scattering are observed by Compact Muon Solenoid (CMS), one of the four detectors of the LHC, located around the interaction points of the collisions. The scattering cross-section can be expressed as a sum in terms of increasing powers of the strong coupling constant $\alpha_{S}$ convoluted with PDFs. The lowest-order $\alpha_{S}^{2}$ term represents the production of two partons in final states whereas terms of higher-order $\alps^3$, $\alps^4$ etc. signify the existence of multi-partons in final states. The highly energetic final state partons emit quarks and gluons with lower energies and give rise to a parton shower (PS). The colored products of parton shower hadronize to a spray of colorless hadrons known as jets. The jets are the final structures observed in the detector. So they carry the significant information of the energy and direction of the initial partons and hence are important to study. The final partons also have the probability to radiate more gluons and quarks which also hadronize and result in multijets in the final state. At LHC, such events are produced in large number and are an important source for testing the predictions given by QCD. They also serve as an important background in the searches for new particles and physics beyond SM. 

The inclusive multijet event cross-section $\sigma_{i-jet}$, given by the process ${{\rm pp}\to i{\rm-}jets~\plus X}$ with every jet counted, is proportional to $\alpha^{i}_{s}$. The study of inclusive jet cross-sections in terms of jet transverse momentum \pt and rapidity $y$ is very important because it provides the essential information about the PDFs and the precise measurement of \alps. Also the ratio of cross-sections given by Eq.~\ref{eq:ratio_mn} is proportional to the QCD coupling constant \alps and hence can be used to determine the value of \alps. 

\begin{equation}
R_{mn} = \frac{\sigma_{m-jet}}{\sigma_{n-jet}} \propto \alps^{m-n}
\label{eq:ratio_mn}
\end{equation}

Instead of studying inclusive cross-sections, the cross-section ratio is more useful because of the partial or complete cancellation of many theoretical and experimental uncertainties between numerator and denominator. The CMS Collaboration has previously measured the ratio of the inclusive 3-jet cross-section to that of the inclusive 2-jet as a function of the average transverse momentum, $<p_{T1,2}>$, of the two leading jets in the event at 7 TeV \cite {Chatrchyan:2013txa}. This study leads to an extraction of \alpsmz = 0.1148 $\pm$ 0.0055, where the dominant uncertainty stems from the estimation of higher-order corrections to the next-to-leading order (NLO) prediction. In this thesis, a measurement of inclusive 2-jet and 3-jet event cross-sections as well as ratio of 3-jet event cross-section over 2-jet \rations, is performed using an event sample collected during 2012 by the CMS experiment at the LHC and corresponding to an integrated luminosity of 19.7\fbinv of pp collisions at a center-of-mass energy of 8\TeV. The event scale is chosen to be the average transverse momentum of the two leading jets, referred to as \httwo in this thesis. The strength of the strong force, \alps at a given energy scale $Q$ is not predicted and has to be extracted from the experiment. Hence, the measurements performed in this thesis are used to extract the value of the strong coupling constant at the scale of the $Z$ boson mass \alpsmz. The value of \alps depends on the energy scale $Q$ and it decreases with the increase of $Q$ scale. The running of \alps with scale $Q$ is also studied and compared with other CMS measurements as well as results from different experiments. This checks the consistency with QCD via the renormalization group equation (RGE)\footnote{According to the RGE, the strong force becomes weaker at short distances corresponding to large momentum transfers. This is referred to a property of QCD called asymptotic freedom.}, which precisely describes the evolution of \alps at the renormalization scale of QCD. 

The organization of this thesis\footnote{The common unit convention based on International System of Units (SI) as followed in particle physics will be used throughout the thesis. In addition, the units electron volt (eV) and barn (b) are used for energy and interaction cross-section, respectively. The reduced Planck constant ($\hbar$) and speed of light ($c$) are set to unity, i.e. $\hbar$ = c = 1.} is as follows :

{\bf Chapter~\ref{chap:Theory}} gives a brief overview of the Standard Model of particle physics and the theory of strong interactions QCD, theory of hadron collisions as well as formation of jets and jet algorithms. 

{\bf Chapter~\ref{chap:Detector}} deals with experimental apparatus which covers the details of the geometry of the CMS detector and its various sub-detectors.

{\bf Chapter~\ref{chap:Reconstruction}} describes the methods of event generation used in different Monte-Carlo event generators, detector geometry simulation and reconstruction of the particles in the detector. This chapter also gives the details of the different approaches of jet reconstruction at CMS and applied jet-energy corrections along with the description of the software framework used in the analysis presented in the current thesis.

{\bf Chapter~\ref{chap:Measurement}} presents the measurement of differential inclusive multijet event cross-sections and the cross-section ratio. The measurements are corrected for detector effects by unfolding procedure which is discussed in details in this chapter. The sources of the experimental uncertainties are studied in details. 

{\bf Chapter~\ref{chap:Theory_Predictions}} contains a detailed description of the NLO perturbative QCD theory predictions obtained using different PDF sets compared to the data and the extraction of \alps. The NLO predictions are corrected with the non-perturbative and electroweak corrections. The theoretical uncertainties are calculated from various sources. At the end of this chapter, the unfolded measurements are compared with the predictions at NLO in pQCD as well as with the predictions from several Monte Carlo event generators.

{\bf Chapter~\ref{chap:Alphas}} describes the method to extract the strong coupling constant at the scale of mass of $Z$ boson \alpsmz from the current measurements and the running of \alps with energy scale Q is presented along with the previous measurements from different experiments.

{\bf Chapter~\ref{chap:Summary}} summarizes the results and conclusions of the work done in this thesis.

{\bf Chapter~\ref{chap:Hardware}} mentions the participation in other hardware and software activities.
