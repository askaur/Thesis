\chapter{Introduction}
\label{chap:Introduction}

Particle physics deals with the study of the basic constituents of matter and the forces governing the interactions among them. The Standard Model (SM) is the most accepted theory which describes the nature and properties of the fundamental particles and their interactions. The elementary particles leptons and quarks, known as fermions, interact through the exchange of gauge bosons and acquire mass through a scalar boson called the Higgs. The four fundamental forces of interaction existing in nature are : the electromagnetic force, the strong force, the weak force and the gravitational force. Quantum Choromodynamics (QCD) is the theory of the strong interactions between the quarks mediated by the massless gluons. The partons (quarks and gluons) have a peculiar property of ``color'' charge. The quarks strongly binds into colorless particles called hadrons such as protons and neutrons together known as nucleons, pions etc. The structure and the properties of sub-atomic particles can be explored by first accelerating them using particle accelerators and then colliding at very high energies. The end products of these collisions get detected in the particle detectors constituting the real data. The data sets analyzed in details to reveal the structure and characteristic properties of the fundamental particles.

To investigate the very rare particles or to search for physics beyond SM, the particle accelerators have become ever bigger and more complex. The Large Hadron Collider (LHC) is one of the today's biggest and most powerful collider where the protons are accelerated and collided at extremely high center-of-mass energies to probe their internal structure which is described by parton distribution functions (PDFs). The PDF sets give the probability to find a parton at an energy scale Q with a fractional momentum $x$ of the proton. Since the proton is not elementary and is made up of partons, the proton-proton (pp) collisions are viewed as interactions between their constituent partons. The final products of the scattering are observed by Compact Muon Solenoid (CMS), one of the detectors located around the interaction points. The scattering cross-section can be written as a sum of terms with increasing powers of the strong coupling constant $\alpha_{S}$ convoluted with PDFs. The lowest-order $\alpha_{S}^{2}$ term represents the production of two-parton final states. Terms of higher-order $\alps^3$, \ldots in the expansion signify the existence of multi-parton final states respectively. The final state partons give a parton shower (PS) due to decrease in energy through emission of other quarks and gluons. The colored products of parton shower hadronize to a spray of colorless hadrons known as jets. The jets are the final structures observed in the detector, so they preserve the energy and direction of the initial partons. Hence the topologies of the initial parton system are assumed to be depicted by those of the final jet system.

The inclusive jet event cross section $\sigma_{i-jet}$ given by ${{\rm pp}\to i~jets~\plus X}$, where every jet is counted, is proportional to $\alpha^{i}_{s}$. So inclusive jet cross-section as a function of jet \pt and rapidity $y$ is one of the important observables as it provides the essential information about the PDFs and the precise measurement of \alps. The final partons have the probability to radiate more gluons resulting in multijets in the final state after hadronization. The events containing multiple jets in the final state are plentiful and provide a fertile testing ground for QCD. They are often an important background in searches for new particles and new interactions at high energies. Instead of studying inclusive cross-sections, it is useful to consider their ratios because many theoretical and experimental uncertainties may cancel between numerator and denominator. The ratio of cross-sections is defined as :

\begin{equation}
R_{mn} = \frac{\sigma_{m-jet}}{\sigma_{n-jet}} \propto \alps^{m-n}
\end{equation}

This quantity is in leading order proportional to the QCD coupling \alps and can be used to determine the value of \alps from jet rates. The CMS Collaboration has previously measured the ratio of the inclusive 3-jet cross-section to the inclusive 2-jet cross-section as a function of the average transverse momentum, $<p_{T1,2}>$, of the two leading jets in the event at 7 TeV \cite {Chatrchyan:2013txa} and lead to an extraction of \alpsmz = 0.1148 $\pm$ 0.0055, where the dominant uncertainty stems from the estimation of higher-order corrections to the NLO prediction. In this analysis, a measurement of inclusive 2-jet and 3-jet event cross-sections as well as ratio of 3-jet event cross-section over 2-jet \ratio, is presented using an event sample collected by the CMS experiment during 2012 at the LHC and corresponding to an integrated luminosity of 19.7\fbinv of pp collisions at a centre-of-mass energy of 8\TeV. The event scale is chosen as before to be the average transverse momentum of the two leading jets, but will be referred to as \httwo in this thesis. The measurements are used to determine the value of the strong coupling constant at the scale of the Z boson mass \alpsmz and the running of \alps with energy scale Q is studied.

This thesis is organized as :

{\bf Chapter~\ref{chap:Theory}} gives a brief overview of the Standard Model of particle physics and the theory of strong interactions QCD with main emphasis on the jets and jet algorithms. 

{\bf Chapter~\ref{chap:Detector}} deals with experimental apparatus which covers the details of the CMS detector and its various sub-detectors.

{\bf Chapter~\ref{chap:Measurement}} presents the measurement of inclusive differential multijet cross-sections and the cross-section ratio. The measurements are corrected for detector effects by unfolding procedure which is discussed in details in this chapter. The sources of the experimental uncertainties are studied in details. 

{\bf Chapter~\ref{chap:Theory_Predictions}} contains a detailed description of the NLO pQCD theory predictions used for comparison with data and the extraction of \alps. The NLO calculations are corrected with the non-perturbative and electroweak corrections. The theoretical uncertainties are calculated from various sources. The unfolded measurements are compared to the predictions at NLO in pQCD and additionally as well as to predictions from several Monte Carlo event generators.  

{\bf Chapter~\ref{chap:Alphas}} describes the method to extract \alpsmz from the current measurement and the running of \alps with energy scale Q is presented along with the previous measurements from different experiments.

{\bf Chapter~\ref{chap:Summary}} summarizes the results and conclusions of the work done in this thesis.
