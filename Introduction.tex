\chapter{Introduction}
\label{chap:Introduction}

Chapter~\ref{chap:Theory}
Chapter~\ref{chap:Detector}
Chapter~\ref{chap:Measurement}
Chapter~\ref{chap:Theory_Predictions}
Chapter~\ref{chap:Alphas}
Chapter~\ref{chap:Summary}
\begin{comment}
Particle physics is the study to understand the basic elements of matter and the forces governing the interactions among them. The Standard Model is the theory which describes the role of the fundamental particles and their interactions. The stable particles which constitute the matter are the leptons (electron and neutrino), the gauge boson (photon) and the nucleons (proton and neutron). Leptons are the basic fundamental particles and do not posssess any substructure ~\cite{per}.

Experiments at particle accelerators such as LEP (Large Electron collider) collide sub-atomic particles at very high energies and reveal their structures and properties. The accelerators produce interactions which are 
observed by detectors. The end products of such interactions are registered in the sophisticated particle detectors, constituting the real data. Detailed studies related to the nature of the produced particles and their characteristic properties can then be made by analysing these data sets. In the Monte Carlo world, the role of machines is played by the event generators. 

Quantum Choromodynamics (QCD) is the currently acccepted theory of the strong interaction between the particles known as partons, classified as quarks q and gluons g, that carry ``color''. These partons get detected in detectors as spray of particles called ``jets''. The dominant process is jet production by scattering of the elementary quark and gluon constituents of the incoming hadronic beams. There are two partons in the initial and final states in the leading order (LO) QCD. The next-to-leading order (NLO) QCD also includes jets from final state radiation (FSR). A jet is associated with the energy and momentum of each final state parton. 
The present work is based on the jet physics. The proton-proton collisions are viewed as the interactions between
 their individual partons. The soft interactions lead to small momentum transfers whereas large momentum is transferred in the hard processes. The structure of jets can be studied theoretically and experimentally.
In this work, the internal structure of jets is studied at LHC energies by measuring the subjet multiplicities in the jets in QCD 2$\rightarrow$2 hard processes. The stucture of a jet depends on type of its origin i.e. primary parton : quark/anti-quark and gluon. In QCD, the coupling strengths of the gluons and quarks are different due to their different color charges. The color factor C$_{A}$ = 3, gives the relative probability for a soft gluon to couple with another gluon and C$_{F}$  = 4/3, determines the corresponding probability for a
soft gluon to couple with a quark. The color factor ratio C$_{A}$/C$_{F}$ = 9/4 implies that the gluon jets have more particles, possess softer momentum as compared to quark jets. As a gluon radiates more than a quark, the number of subjets within a gluon jet are more than that in quark jet. Hence the subjet multiplicity given by the number of subjets resolved within a jet, is a useful variable to differentiate the gluon and quark jets. The ratio r = $\frac{<M_{g}> - 1}{<M_{q}> - 1}$, where $<M_{g}>$ and $<M_{q}>$ are the average subjet multiplicities in gluon and quark jets respectively, gives the comparison of average number of subjets emitted in a gluon jet to that in quark. The value of r = 1 implies that there is no difference between gluons and quarks whereas the value other than 1 expresses the differences between them. 
The organisation of the dissertation is as follows:\\
$\bf {Chapter 2}$ covers a brief overview of the Standard Model, the theory of QCD, explanation of jets.\\
$\bf {Chapter 3}$ presents a brief overview of a prototype detector and various subdetectors which are used for different purposes.\\
$\bf {Chapter 4}$ deals with the description of event generators and Monte Carlo simulation used in present analysis. \\
$\bf {Chapter 5}$ contians the first experimental results on the measurement of subet multiplicities in proton-proton dijet events at a center of mass-energies 7 TeV and 10 TeV. The analysis is performed by the study of jets formed using different jet algorithms.\\
$\bf {Chapter 6}$ summarizes the results and conclusions of the analysis.\\
\end{comment}
