\chapter{Measurement of the Inclusive Differential Multijet Cross Section}
\label{chap:measurement}
The inclusive differential multijet cross sections are measured as a
function of the average transverse momentum, $\httwo =
\frac{1}{2}(\ptone + \pttwo)$, where \ptone and \pttwo denote the
transverse momenta of the two leading jets. \\

\section{Cross Section Definition}
The inclusive mutijet event yields are transformed into a differential cross section which is defined as :
%
\begin{equation}
  \label{inclusive_formula}
  {\dd{\sigma}{\big(\httwo\big)}} = \frac{1}{\epsilon~\lumi_{\mathrm{int,eff}}}\frac{N_\mathrm{event}}{\Delta\big(\httwo\big)}
\end{equation}
%
where $N_\mathrm{event}$ is the number of 2- or 3-jet events counted in an
\httwo bin, $\epsilon$ is the product of the trigger and jet selection
efficiencies, which are greater than 99\%,
\lumins$_{\mathrm{int,eff}}$ is the effective integrated luminosity,
, and $\Delta\big(\httwo\big)$ are the bin widths. The
measurements are reported in units of (pb/\GeV).

For inclusive 2-jet events
sufficient data are available up to $\httwo = 2\TeV$, while for
inclusive 3-jet events (and the ratio \ratio) the accessible range in
\httwo is limited to $\httwo < 1.68\TeV$. In the following, results
for the inclusive 2-jet and 3-jet event selections will be labelled as
$\mathrm {n_{~j}} \geq 2$ and $\mathrm {n_{~j}} \geq 3$, respectively.

\section{Data Samples}
During 2012, CMS collected data at the center of mass energy $\sqrt{s}$ = 8 TeV in four periods A, B, C, D. The datasets are divided into 
samples according to the run period. For run B-D, the \texttt{JetMon} stream datasets contain prescaled low trigger threshold paths (HLT
PFJet40, 80, 140, 200 and 260) while the \texttt{JetHT} stream datasets contain unprescaled high threshold trigger paths (HLT PFJet320 and 
400). For run A, the \texttt{Jet} stream contains all the above mentioned trigger paths. The datasets used in the current study 
are mentioned in the Table~\ref{tab:dataset} along with the luminosity of each dataset : 

\begin{table}[!htbp]
\centering
\caption{Datasets used along with the corresponding run numbers and luminosity.}
\label{tab:dataset}
\vspace{2mm}
\begin{tabular}{cccc}
\hline\hline
   
Run  & Run Range       &  Dataset                               & Luminosity      \rbthm\\\hline
A    & 190456-193621   & /Jet/Run2012A-22Jan2013-v1/AOD         & 0.88 {\fbinv}   \rbtrr\\
B    & 193834-196531   & /Jet[Mon,HT]/Run2012B-22Jan2013-v1/AOD & 4.49 {\fbinv}   \rbtrr\\
C    & 198022-203742   & /Jet[Mon,HT]/Run2012C-22Jan2013-v1/AOD & 7.06 {\fbinv}   \rbtrr\\
D    & 203777-208686   & /Jet[Mon,HT]/Run2012D-22Jan2013-v1/AOD & 7.37 {\fbinv}   \rbtrr\\
\hline\hline
\end{tabular}
\end{table}

The data sets have the LHC luminosity increasing with period, full data sample of 2012 corresponds to an integrated luminosity of 19.7 {\fbinv}. 

\subsection{Monte Carlo samples}
To have a comparison of data results with the simulated events, the \MadGraphF~\cite{bib:madgraph5} Monte-Carlo event generator has been 
used. The \MadGraphF generates matrix elements for High Energy Physics processes, such as decays and $2 \rightarrow n$ scatterings. The 
underlying event is modeled using the tune \Ztwostar. It has been interfaced to \PYTHIAS~\cite{Sjostrand:2006za} by the LHE event record, 
which generates the rest of the higher-order effects using the Parton Showering (PS) model. Matching algorithms ensure that no double-
counting occurs between the tree-level and the PS-model-generated partons. The MC samples are processed through the complete CMS detector 
simulation to allow studies of the detector response and compare to measured data on detector level.

The cross section measured as a function of the transverse momentum \pt or the scalar sum of the transverse momentum of all jets \HT falls 
steeply with the increasing \pt. So in the reasonable time, it is not possible to generate a large number of high \pt events. Hence, the 
events are generated in the different phase-space region binned in \HT or the leading jet \pt. Later on, the different phase-space regions 
are added together in the data analyses by taking into account the cross section of the different phase-space regions. The official CMS 
\MadGraphF + \PYTHIAS MC samples used in this analysis were generated as slices in the \HT phase-space :
\begin{center}
\begin{itemize}
\item The \MadGraphF + \PYTHIAS MC : /QCD\_HT-xxxtoxxxx\_TuneZ2star\_8TeV-madgraph-pythia6/\\Summer12\_DR53X-PU\_S10\_START53\_V7A-v1/AODSIM
\end{itemize}
\end{center}

