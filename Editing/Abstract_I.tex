\begin{center}
{\bf \huge Abstract}
\end{center}

The hadrons colliding at very high center-of-mass energies provide a direct probe to the nature of the underlying parton-parton scattering physics. The scattering of the elementary quarks and gluons, constituents of the incoming hadron beams, produces a high momentum partons which then fragment and hadronize producing a spray of particles. These particles get clustered in the form of jets. The jets being the final structures observed in the detector, preserve the energy and direction of the initial partons. Hence jets can serve as a direct test of theory of strong interactions called Quantum Chromodynamics. The inclusive multijet production cross-section is an important observable which provides the details of parton distribution functions of the colliding hadrons and the precise measurement of the strong coupling constant \alps. Instead of individual cross-sections, the ratio of cross-sections is a better tool to determine the value of \alps as many theoretical and experimental uncertainties cancel in the ratio.

A measurement of inclusive multijet event cross-sections and the cross-section ratio is presented using data from proton-proton collisions collected with the CMS detector at a center-of-mass energy of 8 TeV corresponding to an integrated luminosity of 19.7 \fbinv. Jets are reconstructed with the anti-k$_t$ clustering algorithm for a jet size parameter $R$ = 0.7. The inclusive 2-jet and 3-jet event cross-sections as well as the ratio of the 3-jet over 2-jet event cross-section (\ratio) are measured as a function of the average transverse momenta \pt of the two leading jets in a phase space region ranging up to jet \pt of 2.0 TeV and an absolute rapidity of $|y|$ = 2.5. The measurements after correcting for detector effects are well described by predictions at next-to-leading order in perturbative quantum chromodynamics and additionally are compared to several Monte Carlo event generators. The strong coupling constant at the scale of the $Z$ boson mass is extracted from a fit of the measured \ratio which gives $\alpha_s(M_Z) = 0.1150\,\pm0.0010\,\textrm{(exp)}\,\pm0.0013\,\textrm{(PDF)}\, \pm0.0015\,\textrm{(NP)}\,^{+0.0050}_{-0.0000}\,\textrm{(scale)}$ using MSTW2008 PDF set. The current measurement agrees well with the world average value of \alpsmz = $0.1181 \pm 0.0011$ as well as previous measurements.
\begin{comment}

The collisions of hadrons at very high centre-of-mass energies provide a direct probe to the nature of the underlying parton-parton scattering physics. The scattering of the elementary quark and gluon constituents of the incoming hadron beams produces a spray of particles which are clustered in the form of jets. Jet production at hadron colliders provides an excellent opportunity for testing the theory of strong interactions called Quantum Chromodynamics. The inclusive multijet production cross-section is an important observable which provides the details of parton distribution functions (PDF) of the colliding hadrons and the precise measurement of the strong coupling constant \alps. Instead of individual cross-sections, the cross-section ratio is a better tool to determine the value of \alps as many theoretical and experimental uncertainties cancel in the ratio.

A measurement of inclusive multijet event cross-sections and the cross-section ratio is presented using data from proton-proton collisions collected at a centre-of-mass energy of 8 TeV with the CMS detector corresponding to an integrated luminosity of 19.7 \fbinv. Jets are reconstructed with the anti-k$_t$ clustering algorithm for a jet size parameter R = 0.7. The inclusive 2-jet and 3-jet event cross-sections as well as the ratio of the 3-jet over 2-jet event cross-section \ratio are measured as a function of the average transverse momenta \pt of the two leading jets in a phase space region ranging up to jet \pt of 2.0 TeV and an absolute rapidity of $|y|$ = 2.5. The measurements after correcting for detector effects are well described by predictions at next-to-leading order in perturbative quantum chromodynamics and additionally are compared to several Monte Carlo event generators. The strong coupling constant at the scale of the Z boson mass is determined from a fit of the measured \ratio which gives $\alpha_s(M_Z) = 0.1150\,\pm0.0010\,\textrm{(exp)}\,\pm0.0013\,\textrm{(PDF)}\, \pm0.0015\,\textrm{(NP)}\,^{+0.0050}_{-0.0000}\,\textrm{(scale)}$ using MSTW2008 PDF set. The current measurement is in well agreement with the world average value of \alpsmz = $0.1181 \pm 0.0011$ as well as previous measurements.
\end{comment}
