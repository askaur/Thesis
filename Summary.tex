\chapter{Summary}
\label{chap:Summary}
Inclusive multijet production cross-section measured precisely in terms of jet transverse momentum is one of the important observables in understanding physics at hadron colliders. It provides the essential information about the structure of parton through parton distribution functions (PDFs) and the precise measurement of the strong coupling constant \alps. The value of the strong coupling constant at the scale of the Z boson mass \alpsmz can be determined using cross-section ratio instead of individual cross-sections because many uncertainties of theoretical and experimental origin cancel between numerator and denominator which reduces the dependence on PDFs, renormalization and factorization scales, luminosity etc.

In this thesis, a measurement of the inclusive 2-jet and 3-jet event cross-sections as well as the cross-section ratio \ratio has been presented. The data sample has been collected from proton-proton collisions recorded with the CMS detector at a centre-of-mass energy of 8 TeV and corresponds to an integrated luminosity of 19.7\fbinv. The jets are reconstructed with the anti-\kt clustering algorithm for a jet size parameter R = 0.7. The inclusive 2-jet and 3-jet event cross-sections are measured differentially as a function of the average transverse momentum of the two leading jets, referred as \httwo. The ratio \ratio is obtained by dividing the differential cross-sections of inclusive 3-jet events to that of inclusive 2-jet one in each bin of \httwo. An appropriate selection criteria has been designed for choosing the best
events for analysis. The measurements are performed at a central rapidity of $|y|<2.5$ in a range of $0.3 < \httwo < 2.0\TeV$ for inclusive 2-jet event cross-sections and $0.3 < \httwo < 1.68\TeV$ for inclusive 3-jet event cross-sections and ratio \ratio. 

The measured cross-sections after correcting for detector effects by using an iterative unfolding procedure are compared to the perturbative QCD predictions computed, using \NLOJETPP program, at next-to-leading order (NLO) accuracy and complemented with non-perturbative (NP) corrections that are important at low \httwo. The data are found to be well described by NLO calculations. The upwards trend observed in the inclusive 2-jet and 3-jet data at high \httwo in comparison to the prediction at NLO QCD, is explained by the onset of electroweak (EW) corrections in the 2-jet case. For the 3-jet event cross-sections these corrections have not yet been computed yet. In the 3-jet to 2-jet cross-section ratio \ratio, the EW corrections are assumed to cancel. In fact, NLO QCD provides an adequate description of \ratio in the accessible range of \httwo. In contrast, leading order (LO) tree-level Monte Carlo (MC) predictions obtained using \MadGraphF event generator interfaced to \PYTHIAS exhibit significant deviations. The sources of experimental and theoretical uncertainties are studied in details. The experimental uncertainty ranges from 4 to 32\% for inclusive 2-jet event cross-sections, from 4 to 28\% for 3-jet event cross-sections and from 1 to 28\% for cross-section ratio \ratio. It is dominated by the uncertainty due to the jet energy corrections (JEC) at lower \httwo values and by statistical uncertainty at higher \httwo values. The theoretical uncertainty ranges from 3 to 30\% and 5 to 34\% for inclusive 2-jet and 3-jet event cross-sections respectively and from 3 to 11\% for ratio \ratio. The PDF uncertainty derived with the CT10-NLO PDF set is the dominant source of theoretical uncertainty.

The inclusive multijet cross-sections being proportional to the powers of the strong coupling constant \alps ($\sigma_{\rm n\hy jet} \propto\alps^{\rm n}$) are used to extract the value of the strong coupling constant at the scale of the Z boson mass \alpsmz. In cross-section ratio \ratio which proportional to \alps, many uncertainties and PDF dependencies largely cancel and hence becomes the better tool to extract the value of \alpsmz. In this thesis, a fit of the ratio of the inclusive 3-jet event cross-section to that of 2-jet, \ratio in the range $0.3 < \httwo < 1.68\TeV$ using the MSTW2008 PDF set gives : 

$\alpsmz = 0.1150\,\pm0.0010\,\textrm{(exp)}\,\pm0.0013\,\textrm{(PDF)}\,\pm0.0015\,\textrm{(NP)}\,^{+0.0050}_{-0.0000}\,\textrm{(scale)}$ \\ \hspace*{24mm} = $0.1150\,\pm0.0023\,\textrm{(all except scale)}\,^{+0.0050}_{-0.0000}\,\textrm{(scale)}$ \\
Very similar results are obtained using the MMHT2014 PDF set which gives : 

$\alpsmz = 0.1142\,\pm0.0010\,\textrm{(exp)}\,\pm0.0013\,\textrm{(PDF)}\,\pm0.0014\,\textrm{(NP)}\,^{+0.0049}_{-0.0006}\,\textrm{(scale)}$ \\ \hspace*{24mm} = $0.1142\,\pm0.0022\,\textrm{(all except scale)}\,^{+0.0049}_{-0.0006}\,\textrm{(scale)}$\\ 
The equally compatible values of \alpsmz are determined with separate fits to the inclusive 2-jet and 3-jet event cross-sections provided the range in \httwo is restricted to $0.3 < \httwo < 1.0\TeV$. The extracted \alpsmz values in sub-ranges of \httwo are evolved to corresponding \alpsq along with the error bars at different scales Q. The current measurement of \alpsmz and the running of \alpsq as a function of Q is in well agreement within uncertainties with the world average value of $\alpsmz = 0.1181 \,\pm\, 0.0011$~\cite{Patrignani:2016xqp} and already existing determinations performed by the CMS and other experiments.

The inclusion of the EW corrections in inclusive 2-jet event cross-sections become relevant at \httwo beyond 1 TeV. Their availability for 3-jet one and hence cross-section ratio \ratio can improve the precision of the measurement of \alpsmz. Also as the theoretical calculations will be available for inclusive 4-jet event cross-sections, the various cross-section ratios such as $R_{\rm 43} \propto \alps^1$ and $R_{\rm 42} \propto \alps^2$ can be measured to extract the value of the strong coupling constant more precisely. Currently LHC is running at high center-of-mass energy of 13 TeV delivering a higher instantaneous luminosity and this makes possible to access the extended phase space and perform the measurements with more accuracy.
%~\cite{Chatrchyan:2013txa, Chatrchyan:2013haa, Khachatryan:2014waa, CMS:2014mna, ATLAS:2015yaa, Khachatryan:2016mlc} 
